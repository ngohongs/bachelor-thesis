\chapter{Základ teorie mechaniky tekutin}
 
 V této kapitole je shrnutí teorie mechaniky tekutin, která je nezbytnou součástí pro simulování fyzikálně korektního chování vody. Teorie je zde vyložena takovým způsobem, aby byl čtenář s ní seznámem a co nejrychleji pochopil principy chování tekutin, neobsahuje žádné rigorózní vysvětlení problematiky.
 
    \section{Navierovy--Stokesovy rovnice}
    
        Proud tekutin se v reálném světě řídí podle Navierových--Stokesových rovnic (NSE), soustavou nelinearních diferenciálních rovnic:
    
        \begin{equation}
            \nabla \cdot \mathbf{v} = 0 \label{eq:mass_conservation}
        \end{equation}
        \begin{equation}
            \frac{ \partial \mathbf{v}}{\partial t} + \mathbf{v} \cdot \nabla \mathbf{v} + \frac{1}{\rho} \nabla p = \mathbf{g} + \nu \nabla^2 \mathbf{v}, \label{eq:momentum_conservation}
        \end{equation}
    
        \noindent kde vektor $ \mathbf{v} = (u, v, w) $ označuje rychlost tekutiny, $ \rho $ označuje hustotu tekutiny, $ p $ tlak, kterým působí tekutina na své okolí, $ \mathbf{g} = (x, y, z) $ je gravitační zrychlení, $ \nu $ je značení kynematické viskozity tekutiny. Symboly $ \nabla$, $ \nabla \cdot $, $ \nabla^2 $ označují diferenciální operátory nabla, divergence a Laplacův operátor.  
    
            \subsection{Vysvětlení}
                Před vysvětlení jednotlivých rovnic je dobré zmínit, že většina teorie mechaniky tekutin je založena na odvětví matematiky vektrové analýzy a pracují s\,vektorovými poli, resp. skalárními poli. NSE např. pracují s vektorovým poli, resp. se skalárními poli, které jednotlivým bodům prostoru přiřazují vektor určující rychlost proudu tekutiny, resp. hodnotu tlaku.
    
                Na první pohled vypadají rovnice těžce uchopitelné, ale myšlenka za nimi je velmi jednoduchá. První rovnice \ref{eq:mass_conservation} popisuje zákon o zachování hmotnosti, tím je myšleno, že není možné, aby hmota tekutiny na některém místě z ničeho vznikla nebo zanikla.
    
                Druhá rovnice \ref{eq:momentum_conservation} popisuje zákon o zachování hybnosti a je ve své podstatě Newtonův druhý zákon.
                
                \begin{equation}
                    \mathbf{f} = m \mathbf{a} \label{eq:newton_second}
                \end{equation}
    
                \noindent Zrychlení $ \mathbf{a} $ lze přepsat jako derivace rychlosti podle času.

                \begin{equation}
                    \mathbf{f} = m \frac{D \mathbf{v}}{D {t}} \label{eq:first_step},
                \end{equation}

                \noindent kde $\frac{D \mathbf{v}}{D {t}}$ značí tzv. materiálovou derivací. Pro odvození celé rovnice je třeba rozvést sílu $ \mathbf{f} $, které na tekutinu působí\;\cite{Bridson2007}. Jedna z nich je samozřejmě gravitace, jejíž hodnota je vyčíslena jako $ m\mathbf{g} $.
                
                
                Další síly vytváří tekutina sama na sobě. První z nich je síla způsobena rozdílem tlaků v tekutině. Tekutina se v oblastech s vyšším tlakem přesouvá do oblastí s nižším tlakem. Její hodnotu můžeme zapsat jako $ \minus V \nabla p $ \footnote{Operace $\nabla p$ vyjadřuje vektorové pole, ve kterém vektory směřují, z jednotlivých bodů na své sousední body, aby jeho hodnota ve skalárním poli nejrychleji vzrostla.}. 
        
                Další síla působící na tekutinu je ovliněna její viskozitou. Viskozita působí při každém pohybu částic tekutiny a snaží se vyrovnat rychlost částice rychlostí svých sousedních částic. Její hodnotu můžeme vyjádřit jako\:$  V \mu \nabla^2 \mathbf{v} $ \footnote{Operace $ \nabla^2 \mathbf{v} $ vyjadřuje míru odchýlení rychlosti částice okolo svého okolí.}, kde $ \mu $ označuje koeficient dynamické viskozity.
    
                \begin{equation}
                    m \mathbf{g} - V \nabla p + V \mu \nabla^2 \mathbf{v} = m \frac{D \mathbf{v}}{Dt} \label{eq:second_step}
                \end{equation}
    
                Rovnice \ref{eq:newton_second}, \ref{eq:first_step} a \ref{eq:second_step} předpokládají, že tekutinu lze rozložit na konečně mnoho malých částí, tímto způsobem je do vyčíslení představena výpočetní chyba. Řešením toho problému je celou rovnici \ref{eq:second_step} vydělit $V$, aby zachytila pohyb nekonečně mnoho nekonečně malých částic tekutin. Dobré je připomenout, že hustota $\rho$ lze vyjádřit jako $\frac{m}{V}$.
    
                \begin{equation}
                    \rho \mathbf{g} - \nabla p + \mu \nabla^2 \mathbf{v} = \rho \frac{D \mathbf{v}}{D \mathbf{t}}  
                \end{equation}
    
                Následně po vydělení hustoty $\rho$, vyjádření derivace $\frac{D \mathbf{v}}{ Dt}$ podle pravidla pro složené funkce a prohození sčítanců získáme druhou Navierovu--Stokesovu rovnici. Kynematická vyskozita $\nu$ je rovna $\frac{\mu}{\rho}$.
    
                \begin{equation}
                    \frac{\partial \mathbf{v}}{\partial t} + \mathbf{v} \cdot \nabla \mathbf{v} + \frac{1}{\rho} \nabla p = \mathbf{g}  + \nu \nabla^2 \mathbf{v} 
                \end{equation}
                
        \newpage
                
        \section{Využití}
            Výhodou NSE je, že je lze aplikovat na téměř jakékoliv tekutiny. V praxi se využívají pro modelování počasí, podnebí pro předpověď počasí, proudění v\,oceánech, výpočtu aerodynamických vlastností vozidel. Přestože využití rovnic je nespočetné, mají zásadní problém, neboť do dnes nevíme, zdali mají pro náhodný vstup nějaké řešení. Pro zjednodušení výpočtu existují několik aproximací, které je umožňují aplikovat se zanedbatelnými chybami v modelovacích systémech.
            
        \section{Popisy tekutiny}

            V teorii mechaniky tekutin existují dva různé pohledy popisu tekutin. Na základě těchto popisů jsou následně založeny algoritmy pro simulaci kapalin, resp. plynů, které jich využívají pro diskretizaci NSE.


            \subsection{Lagrangeův popis}

                Lagrangeův popis je jeden z popisů kontinua, který si většina nejspíše vybaví. Na tekutinu nahlíží jako na systém částic. Na jednotlivé body tekutiny nahlíží jako částice, které mohou být na základě potřeby různě velké, např. jako molekule nebo části tekutiny o nějakém objemu. Ke každé z nich Lagrangeův popis přiřazuje její pozici $\mathbf{x}$ a rychlost $\mathbf{v}$ a sleduje je, jak se tyto hodnoty v\,čase mění.

            \subsection{Euleurův popis}
            
                Euleurův popis kontinua je na první pohled neintuitivní. Eulerův postup diskretizuje prostor tekutiny na pevně dané oblasti, ve kterých měří vlastnosti tekutiny, např. tlak, rychlost proudu, teplotu, jak se v čase mění. 
                
                Přestože tento postup vypadá omezující a složitý kvůli sledování veličin tekutin jen v pevně daných bodech, je v metodách pro simulování tekutin preferovaným popisem kontinua. Hlavní výhodou Eulerova pohledu je jednoduchost výpočtu prostorových derivací jako $\nabla p $ nebo $ \nabla \mathbf{v}$, které se lépe aproximují v\,pevné Eulerově mřížce oproti shlukům pohybujcích se částic \cite{Bridson2007}.

            

 
 

\chapter{Vlastnosti vodního povrchu}

    Tato kapitola obsahuje popis vizuálních vlastností vodní hladiny, které v reálném světě lze zpozorovat, a k jejím hlavním rysům jsou vypsané algoritmy, které je simulují. Z\,pohledu počítačové grafiky lze vlastnosti vody rozdělit do dvou kategorií:

    \begin{description}
        \item[dynamické vlastnosti] popisující pohyb vodní hladiny,
        \item[světelné vlastnosti] popisující interakce povrchu vody se světelnými prsky.
    \end{description}

    \section{Dynamické vlastnosti}

        Dynamické vlastnosti zachycují, jak se vodní hladina pohybuje a jak reaguje na dynamické prostředí. V současnosti nejrealističtější výsledků je dosaženo simulací Eulerovy vody (nahlíží na vodu Eulerovým popisem), jejíž prostor je rozdělen do alespoň 512\textsuperscript{3} buněk. Takto řídké rozdělení obsahuje přes 100 milion neznámých pro vyřešení a s užitím globálních zobrazovacích metod jako ray-tracing pro realistickou vizualci odrazů, refrakcí a kaustik je výpočetně nemožné simulaci provést v reálném čase \cite{Mueller2008}.


        \begin{figure}\centering
            \includegraphics[width=0.5\textwidth]{img/Guendelman2005}
            \caption{Ukázka Guendelmanovy off-line simulace Eulerovy tekutiny \cite{Guendelman2005}}\label{fig:guendelman-offline}
        \end{figure}

        Proto real-time simulace vody vhodné pro aplikace jako hry musí nutně splňovat tyto podmínky \cite{Mueller2008}:

        \begin{itemize}
            \item být výpočetně rychlé -- zlomek 15 ms, který je třeba pro vykreslení jednoho snímku,
            \item být paměťově nenáročné,
            \item být stabilní -- korektně reagovat i na nefyzikálně pohybující se objekty.
        \end{itemize}

        \newpage 

        Jeden přístup, jak se nejvíce příblížit off-line simulaci a stále dodržet podmínky pro real-time simulaci ve hrách, je zachovat stejný algoritmus jako při off-line simulaci, ale zmenšit rozlišení prostoru simulace. Tento postup ale spíše ubírá na realitě vodního povrchu, neboť se voda nefyzikálně shlukuje a detaily z obrazu jsou vynechány.

        Další z možností, jak splnit výše zmíněné podmínky, je omezit míru interakce s vnějším prostředím, která tvoří největší výpočetní překážku. Na základě významnosti vodních útvarů ve scéně může simulace reagovat na všechny rigidní tělesa nebo u případů, ve kterých voda slouží jako pozadí, opominout jakoukoliv interakci. Podle tohodle principu můžeme dělit metody simulace vodní plochy na \cite{Mueller2008}:

        \begin{itemize}
            \item procedulární
            \item částicové
            \item hybridní
        \end{itemize}

            \subsection{Procedulární metody}

                Procedulární metody simulují konečné efekty jako vlnění vodního povrchu, které nejsou vyvolány fyzikální činitely. Největší výhodou této metody je výpočetní rychlost a versetalita, na druhou stranu ale vodní útvary nereagují na dynamické prostředí.

                Na základě tichto vlastností se procedulární metody používají pro vizualici rozsáhlých vodních ploch, které ve scéně hrají malou roli, např. jako oceán v\,pozadí scény.

                \newpage
                
                \subsubsection{Simulace podle sinusoid}
                    Mezi prvními, kteří zkoumali procedulární metody pro zobrazení vodního povrchu, byl Nelson L. Max ve své práci \uv{Vectorized Procedural Models for Natural Terrain: Wave and Islands in the Sunset,} ve které modeluje vlnění povrchu pomocí sinusoid. Na základě horizontální pozici a času manipuloval výšku vrcholů roviny \cite{Max1981}. Výšku jednotlivých vrcholů roviny vyjádřil jako funkci

                    \begin{equation}
                        h(x, z, t) = -y_0 + A \sin (k_{x}x+k_{z}z-\omega t + \varphi), \label{eq:sin_wave}
                    \end{equation}

                    \noindent kde $(x, z)$ je horizontální pozice vrcholu roviny (v celé práci se bude považovat, že kladná osa y směřuje nahoru), $t$ je čas, $y_0$ je výška hladiny vody v klidovém stavu (při nulovém výskytu vln), $A$ reprezentuje amplitudu roviny, $\mathbf{k} = (k_x, k_z)$ je vlnový vektor reprezentující směr a rychlost propagace vln, $\omega$ je úhlová frekvence a $\varphi$ je fáze vlny.

                    \begin{figure}\centering
                        \includegraphics[width=0.5\textwidth]{img/sin_wave}
                        \caption{Simulace vlnění pomocí jedné sinusoidy}\label{fig:3d_sin_wave}
                    \end{figure}
                    
                    Tento model se ale vlní jen v jednom směru a výsledné vlny vypadají nerealisticky hladce. K dosažení větších detailů lze k funkci přičíst další sinusoidy s odlišnými parametry amplitudy, vlnového vektoru nebo úhlové frekvence:  

                    \begin{equation}
                        h(x, z, t) = -y_0 + \sum_{i=1}^{N_w} A_i \sin (k_{x_i}x+k_{z_i}z-\omega_i t + \varphi_i), \label{eq:sin_wave_sum} 
                    \end{equation}

                    \noindent kde $N_w$ je celkový počet vln.

                    \begin{figure}\centering
                        \includegraphics[width=0.5\textwidth]{img/sin_wave_sum}
                        \caption{Simulace vlnění pomocí sčítání sinusoid}\label{fig:3d_sin_wave_sum}
                    \end{figure}
                    
                    \newpage 
                    
                    Realistického řešení však za pomocí jen obyčejných sinusoid nelze dosáhnout. Max si všiml, že vlnění oceánu s vyššími amplitudami mají užší hřeben a mělčí údolí, zatímco vrchol a údolí sinusoidy je stejně oblý. Mark Finch a Cyan Worlds přišly s řešením toho problém, které stále využívá jednoduchých sinusoid. Kromě obyčejných funkcí sinusoid přičítá navíc funkci
                    
                    \begin{equation}
                        f_i(x,z,t) = -y_0 + 2 \left( \frac{\sin (k_{x_i} x + k_{z_i} z - \omega_i t + \varphi_i) + 1}{2} \right)^k, \label{eq:sin_better}
                    \end{equation}
                    
                    \noindent kde $ k \in \mathbb{R}^+ $ určuje míru, jak má být hřeben úzký \cite{Fernando2004}.
                    
                    \begin{figure}\centering
                        \includegraphics[width=\textwidth]{img/sin_better}
                        \caption{Porovnání mezi vlnami funkcí \ref{eq:sin_wave} a \ref{eq:sin_better}}\label{fig:sin_better}
                    \end{figure}
                
                \subsubsection{Simulace podle Gerstnerovy vlny}
                    Limitující faktor metody založené na transformaci vrcholů pomocí sinusoid je, že manipuluje s jediným parametrem vrcholu, tj. výškou, a pro realističtější výsledky bylo třeba manipulovat vrcholy i v horizontální rovině. Tento problém řeší cyklické křivky, tzv. trochoidy \footnote{Trochoidou nazýváme trajektorii bodu pevně spojeného s \uv{kotálející se} kružnice po nehybné přímce \cite{Cycle2022}.}, podle kterých Franz Josef Gerstner, německý fyzik, modeloval vlnění hladiny v hlubokých vodách \cite{Gerstner1804}. 
                    
                    \begin{figure}\centering
                        \includegraphics[width=0.75\textwidth]{img/trochoidal_wave}
                        \caption{Gerstnerova vlna s fázovou rychlostí $ c $, vlnovou délkou $ \lambda $ a rozdílem hřebenu a údolí $ H $ \cite{Kraaiennest2015}}
                        \label{fig:trochoidal_wave}
                    \end{figure}
                    
                    Dnes v teorii dynamiky tekutin vlnění založené na trochoidních křivkách nazýváme Gerstnerovými vlnami. Mezi prvními, kteří se Gerstnerovými vlnami zabývali, byli Alain Fournier a William T. Reeves, kteří transformovali vrcholy roviny podle tichto parametrických rovnic \cite{Fournier1986}:
                   
                    \begin{equation}
                     x = x_0 + r \cos ( k_x x_0 + k_z z_0 - \omega t )
                    \end{equation}
                    
                    \begin{equation}
                     y = r \sin ( k_x x_0 + k_z z_0 - \omega t  )
                    \end{equation}

                    \begin{equation}
                     z = z_0 + r \cos ( k_x x_0 + k_z z_0 - \omega t  ),
                    \end{equation}
                    
                    \noindent kde vektor $(x_0, z_0)$ reprezentuje klidovou pozici vrcholu roviny, $ r $ je délka opisujícího bodu od středu kružnice trochoidní křivky, vektor $\mathbf{k} = (k_x, k_z)$ určuje rychlost a směr propagace vlnění, $ \omega $ označuje úhlovou frekvenci a $ t $ je čas.
                    
                    Hodnota $ s = r|\mathbf{k}| $ určuje, jak strmá bude vlna. Pro hodnutu $ s = 0.2 $ má vlna tvar jako sinusoida, pro $ s = 1 $ má tvar cykloidy a pro $ s > 1 $ dochází samoprotínání trochoidy, a proto by se hodnotám vyšší než jedna pro vizualizaci vodního povrchu mělo vyhýbat. V obrázku \ref{fig:s_comparison} je porovnání hodnot $ s $ ve dvourozměrném prostoru.   

                    \begin{figure}\centering
                        \includegraphics[width=0.75\textwidth]{img/s_comparison}
                        \caption{Porovnání hodnot $ s $ ve dvourozměrném prostoru} \label{fig:s_comparison}
                    \end{figure}
                    
                    Detailnějších výsledků lze získat podobně jako u sinusoid pomocí skládání Gerstnerových vln:
                    
                    \begin{equation}
                     x = x_0 + \sum_{i=1}^{N_w} r_i \cos ( k_{x_i} x_0 + k_{z_i} z_0 - \omega_i t ) \label{eq:gerstner_x}
                    \end{equation}
                    
                    \begin{equation}
                     y = \sum_{i=1}^{N_w} r_i \sin ( k_{x_i} x_0 + k_{z_i} z_0 - \omega_i t  )
                    \end{equation}

                    \begin{equation}
                     z = z_0 + \sum_{i=1}^{N_w} r_i \cos ( k_{x_i} x_0 + k_{z_i} z_0 - \omega_i t  ). \label{eq:gerstner_z}
                    \end{equation}

                    \begin{figure}\centering
                        \includegraphics[width=0.5\textwidth]{img/gerstner_wave}
                        \caption{Gerstnerovy vlny v simulaci vodního povrchu podle tutoriálu od Jaspera Flicka \cite{Gerstner2022}}\label{fig:gerstner_wave}
                    \end{figure}
                \subsubsection{Simulace podle Fourierovy transformace}
                    Předešlé dvě metody jsou kvalitními nástroji pro simulaci vodního povrchu a za jejich pomocí lze i získat fotorealistických výsledků pro vysoký počet vln $N_w$ v řádech tísíců nebo více. Vysoký počet představuje však výpočetní překažku. Obvykle by  výpočet posunutí podle rovnic \ref{eq:sin_wave_sum} nebo \ref{eq:gerstner_x}--\ref{eq:gerstner_z} herní enginy provedly ve \emph{vertex shaderech}, pro které je však výpočet funkcí sinus a cosinus náročné \cite{SinShader2022}.
                    
                    Johanson řešil výpočetní náročnost adaptivní metodou, která na základě vzdálenosti kamery a vodní plochy vyřazuje vlny \cite{Darles2011}. V případě, že je kamera dostatečně vzdálená od vodní plochy, vyřazuje Johanson vlnění s vysokými frekvenci. Efektevnějšího výsledku dosáhl Lee \cite{Darles2011}, který před vykreslení vodní plochy navíc zahazuje vrcholy geometrie hladiny, které jsou mimo rozsah kamery.
                    
                    V současnosti nejlepším řešením tohoto problému je užití rychlé Fourierovy transformce (FFT), resp. inverzní Fourierovy transformace (IFFT) podle Jerryho Tessendorfa \cite{Tessendorf2001}. Sčítání sinusoid v rovnici \ref{eq:sin_wave_sum} je ve podstatě inverzní Fourierova transformace, kde jednotlivé sinusoidy přispívají ke konečnému vlnění. Tesssendorf podle stastických dat vlnění oceánů, např. ze satalitních snímků, modeluje výšku vrcholů hladiny jako inverzní Fourierovu transformaci funkce
                    
                    \begin{equation}
                        h(\mathbf{x},t) = \sum_k \widetilde{h}(\mathbf{k}, t) e^{i \mathbf{k} \cdot \mathbf{x}},
                    \end{equation}
                    
                    \noindent kde $\mathbf{x}$ je pozice vrcholů roviny, funkce $ \widetilde{h} $ závislá na čase $t$ obsahuje informaci o amplitudě a fázy sinusoidy $e^{i \mathbf{k} \cdot \mathbf{x}}$ s vlnovým vektorem $\mathbf{k}$. Tessendorf následně podle oceánografických dat vhodně volí funkci $ \widetilde{h} $, aby bylo vyslédné vlnění nejvíce realistické. 
                    
                    Časová komplexita triviálního sčítání sinusoid v rovnici \ref{eq:sin_wave_sum}, tj. inverzní Fourierovy transformace, je $\mathcal{O} (n^2)$, kdežto užitím algoritmů pro IFFT se redukuje časová komplexita na $\mathcal{O} (n \log n)$  \cite{FFTCompl2022}.
                    
                    \begin{figure}\centering
                        \includegraphics[width=0.5\textwidth]{img/fft_ocean}
                        \caption{Ukázka simulace podle FFT od Asylum Darth \cite{FFTPicture2022}}
                    \end{figure}

            \subsection{Částicové metody}  
                Procedulární metody jsou stavěny pro vizualici rozsáhlých vodních ploch jako oceány a moře. Protože jsou tyto vodní útvary rozlehlé, jakýkoliv zásah buď hráče, či prostředí by měl na výsledné zobrazení hladiny minimální efekt. 
                
                V někteřých případech se ale hladina vody nechová periodicky nebo reakce na dynamické prostředí je právě při zobrazování žádoucí jako u vody fontány nebo kaluží. Tyto detailní vlastnosti vody řeší částicové modely využívající Langrageova popisu kontinua. Částicové metody simulují vodní útvary jako systém částic, které reprezentují určitou hmotu kapaliny. 
                
                Poprvé, kdo využil částicových systémů pro simulaci tekutiny, byl v roce 1983 Reeves \cite{Reeves1983}. Reevesům přístup byl však velmi primitivní oproti dnešním variacím, pohyb částic nekorektně modeloval nezávislé na ostatních. V realitě jeho simulace se více blížila k simulaci shlukům jemných částic, např. prachu.
                
                Pro korektní chování je třeba, aby se jednotlivé částice mezi sebou přitahovaly a odpuzovaly. Obecně lze tuto sílu mezi dvěma částicemi vyčíslit jako 
                
                \begin{equation}
                    f(\mathbf{x}_i, \mathbf{x}_j) = F(|\mathbf{x}_i - \mathbf{x}_j|) \cdot \frac{\mathbf{x}_i-\mathbf{x}_j}{|\mathbf{x}_i-\mathbf{x}_j|},
                \end{equation}

                \noindent kde $\mathbf{x}_i$, resp. $\mathbf{x}_j$ je pozice částice $i$, resp. částice $j$, funkce $F$ je velikost síly \cite{Bridson2007}.
                
                Pro simulaci viskózní tekutin Miller a Pearce modelovali tuto sílu, jako by byly jednoltivé částice spojeny pružinou \cite{MillerPearce1989}. Fyzikálně korektnějšího chování lze získat při volbě funkce $ F $ jako hodnotu Lennardovy--Jonesovy síly, která se používá v simulacích molekulární dynamiky,
                
                \begin{equation}
                    f(\mathbf{x}_i, \mathbf{x}_j) =  \left( \frac{k_1}{|\mathbf{x}_i - \mathbf{x}_j|^m} - \frac{k_2}{|\mathbf{x}_i - \mathbf{x}_j|^n} \right) \cdot \frac{\mathbf{x}_i - \mathbf{x}_j}{|\mathbf{x}_i - \mathbf{x}_j|},
                \end{equation}
                
                \noindent kde $k_1, k_2, m, n$ jsou ovládací parametry. Obvyklou volbou je $k_1 = k_2 = k$, $m = 4$ a $n = 2$ \cite{Bridson2007}.
                
                \newpage
                
                Pro real-time simulace může být tento přístup pomalý. V případě, že vodní útvar obsahuje $n$ částic, musí se funkce $f$ evaluovat $\mathcal{O}(n^2)$. Existují ale optimilizace, které rozdělují částice do pravidelné mřížky, čímž redukují složitost na $\mathcal{O}(n)$. Přestože ji lze vypočítat poměrně rychle, simulace se vůbec neopírá o\,NSE. 
                
                \subsubsection{Simulace podle Smoothed Particle Hydrodynamics}
                    Aktuálně nejrozšířenější částicová metoda založená na výpočtu sil podle NSE vychází z Monaghanova článku \uv{Smoothed Particle Hydrodynamics} \cite{Monaghan2005}. Monaghan vychází interpolační metody Smoothed Particle Hydrodynamics (SPH), kterou z prva Lucy vyvinul pro astrofyzické problémy \cite{Lucy1977}. Lucyho metoda je dostatečně obecná, aby ji bylo možné využít v simulaci tekutin pro výpočet skálárních polí hustoty nebo tlaku.
                
                    SPH lze v počítačové grafice přirovnat ke konvoluci, kde místo barvy pixelů pracuje s vlastnostmi částic. SPH distribuje obecnou veličinu pro danou částici jako sumu vážených hodnot veličin jejích sousedů. Hodnota obecné veličiny $A$ pro částici $i$ je podle SPH \cite{Mueller2003}
                    
                    \begin{equation}
                      A_i(\mathbf{x}_i) = \sum_{j} m_j \frac{A_j}{\rho_j} W(|\mathbf{x}_i-\mathbf{x}_j|, h), \label{eq:sph}
                    \end{equation}

                    \noindent kde $\mathbf{x}_i$, resp. $\mathbf{x}_j$, je pozice částice $i$, resp. $j$, $j$ iteruje přes všechny částice, $m_j$ je hmotnost částice, $A_j$ je hodnota veličny $A$ a $\rho_j$ je hustota pro částici $j$. Funkce $W(|\mathbf{x}_i - \mathbf{x}_j|, h)$ se nazývá \emph{smoothing kernel} s poloměrem $h$, což je analogií jádra u konvoluce. Obrázek \ref{fig:sph}
                    ilustruje, jak \emph{smoothing kernel}, které mají obvykle tvar podobné Gaussovu rozdělení, váží hodnoty částic. Největší váhu $W$ přiřadí vrchlům nejblíže středu, tj. částici $\mathbf{x}_i$, a nejnižší částicím vzdálené od středu délkou $h$. Validní \emph{smoothing kernely} musí navíc splnit normaliziční podmínka \cite{Bridson2007, Mueller2003}:
                    
                    \begin{equation}
                        \int W(r) dr = 1
                    \end{equation}
                
                    \begin{figure}\centering
                        \includegraphics[width=0.5\textwidth]{img/sph}
                        \caption{Ilustrace vážení \emph{smoothing kernelem} \cite{Jlcercos2018}}\label{fig:sph}
                    \end{figure}
                    
                    NSE mimo samotných hodnot veličin dále pracuje s gradientem (operace\;$\nabla$), jehož aplikace na rovnici \ref{eq:sph} ovliňí jen \emph{smoothing kernel} 
                    
                    \begin{equation}
                      A_i(\mathbf{x}_i) = \sum_{j} m_j \frac{A_j}{\rho_j} \nabla W(|\mathbf{x}_i-\mathbf{x}_j|, h),
                    \end{equation}
                    
                    \noindent výpočet Laplace (operace $\nabla^2$) obdobně jako u gradientu působí jen na jádro $W$ 
                   
                    \begin{equation}
                      A_i(\mathbf{x}_i) = \sum_{j} m_j \frac{A_j}{\rho_j} \nabla^2 W(|\mathbf{x}_i-\mathbf{x}_j|, h).
                    \end{equation}
    
                    Obvyklá volba jádra $W$, které lze využít až na dvě vyjímky téměř pro každou interpolaci veličin v simulaci tekutin, je \emph{kernel poly6} \cite{Bridson2007, Mueller2003}
                    
                    \begin{equation}
                        W_{poly6}(r, h) = \frac{315}{64 \pi h^9}
                        \begin{cases}
                            (h^2 - r^2)^3 & 0 \le r \le h \\
                            0 & \text{jindy}
                        \end{cases}.
                    \end{equation}
                    
                    \noindent \emph{Poly6} ale nekoretně aproximuje síly tlakového pole, v simulaci se částice pod vysokým tlakem nefyzikálně shlukují \cite{Mueller2008}, neboť $\nabla W$, jež je nutné pro výpočet síly způsobené rozdílem tlaku (rovnice \ref{eq:momentum_conservation}), se ve středu jádra blíží k nule, což vede k zanadbání odpuzujících sil. Desburn řešil tento problém nahrazením jádra \emph{poly6} za \emph{kernel} \cite{Desburn1996}
                    
                    \begin{equation}
                        W_{spiky}(r, h) = \frac{15}{\pi h^6}
                        \begin{cases}
                            (h - r)^3 & 0 \le r \le h \\
                            0 & \text{jindy}
                        \end{cases}.
                    \end{equation}
                    
                    Poslední nepřesnost jádra \emph{poly6} se vyskytuje při výpočtu silové pole vytvořené viskozitou tekutiny. Výsledek jeho Laplacovy operace, od kterého se odvíjí síla vyvolaná viskozitou (rovnice \ref{eq:momentum_conservation}), se u středu jádra pohybuje v záporných číslech, které může vést k nekorektnímu zvýšení rychlosti částice a následné nestabilititě simulace \cite{Mueller2003}. Na základě tohoto nedostatku Müller používá při výpočtu silového pole vyvolané viskozitou jádro  
                    
                    \begin{equation}
                        W_{viskozita}(r, h) = \frac{15}{2\pi h^3}
                        \begin{cases}
                            -\frac{r^3}{2h^3} + \frac{r^2}{h^2} + \frac{h}{2r} - 1 & 0 \le r \le h \\
                            0 & \text{jindy}
                        \end{cases}.
                    \end{equation}

                    \newpage
                    
                    Pro konečnou simulaci vody je třeba vyřešit soustavu NSE za pomocí pole rychlosti, pole hustoty a pole tlaku, které SPH interpoluje z vlasnost částic. Díky aplikací Langrageova popisu kontinua se některé rovnice výrazně zjednodušší. Rovnice o zachování hmoty \ref{eq:mass_conservation} je automaticky splněna, neboť počet částic zůstane při simulaci konstantní. Dále lze zjednodušit v rovnici o zachování hybnosti \ref{eq:momentum_conservation} výraz  $\frac{\partial \mathbf{v}}{\partial t} + \mathbf{v} \cdot \nabla \mathbf{v}$, protože se částice pohybují s tekutinou, může se z výrazu vynechat člen $\mathbf{v} \cdot \nabla \mathbf{v}$, který reprezentuje změnu rychlosti v případě, že se  částice pohybuje \emph{skrz} tekutinu. \cite{Mueller2003}. Konečná NSE pro SPH částicový systém zní takto 
                    
                    \begin{equation}
                      \rho \frac{\partial \mathbf{v}}{\partial t} = - \nabla p = \rho \mathbf{g} + \mu \nabla^2 \mathbf{v}.
                    \end{equation}

                    
                    
                    Pohyb částíce je následně určený silami vyvolané rozdílem tlaků $-\nabla p$, viskozitou tekutiny $\mu \nabla^2 \mathbf{v}$ a gravitací $\rho \mathbf{g}$ (na vodu působí i extérní síly, ty lze zakomponovat s gravitací do $\mathbf{g}$.
                    
                    Částice si ale udržují informace jen o své pozici, rychlosti a hmotnosti, zbylé vlastnosti si musí simulátor spočítat sám. Každá z nich určuje nějaký objem $V_i = \frac{m_i}{\rho_i} $, zatímco hmotnost zůstává během celé simulace konstantní, $\rho_i$ se musí evaluovat v každé vykreslovací smíčce (viz obrázek \ref{fig:sph_density}) \cite{Mueller2003}. Hustota $\rho_i$ částice $i$ na pozici $\mathbf{x}_i$ je pak podle SPH \ref{eq:sph} určená jako
                    
                    \begin{figure}\centering
                        \includegraphics[width=0.5\textwidth]{img/sph_density}
                        \caption{Demonstrace rozdílů v hustotě tekutiny \cite{SPHNvidia2011}}\label{fig:sph_density}
                    \end{figure}
                    
                    \begin{equation}
                        \rho(\mathbf{x}_i) = \sum_{j} m_j \frac{\rho_j}{\rho_j} W(| \mathbf{x}_i - \mathbf{x}_j|, h) = \sum_{j} m_j W(| \mathbf{x}_i - \mathbf{x}_j|, h).
                    \end{equation}
                    
                    Dále je třeba vypočítat tlak $p_i$ na pozici $\mathbf{x}_i$, který lze aproximovat pomocí stavové rovnice ideálního plynu \cite{Mueller2003}
                    
                    \begin{equation}
                        p_i = k \rho_i, \label{eq:ideal_state}
                    \end{equation}
                    
                   \noindent kde $k$ je konstanta plynu závislá na teplotě a $\rho_i$ hustota částice. Desburn navrhl úpravu rovnice \ref{eq:ideal_state}, která přispěla k větší stabilitě simulace \cite{Desburn1996}
                   
                    \begin{equation}
                        p_i = k (\rho_i - \rho_0), 
                    \end{equation}
                   
                   \noindent kde $\rho_0$ označuje klidovou hustotu tekutiny. Protože síla vyvolaná rozdílem tlaku závisí na gradientu tlaku, je posun všech hodnot o konstantu stále validním řešením.
            
                   S výše uvedenými informacemi lze nakonec spočítat síly působící v tekutině $- \nabla p$ a $\mu \nabla^2 \mathbf{v}$. Na základě SPH \ref{eq:sph} je tlaková síla působící na částici $i$ rovna
                   
                   \begin{equation}
                        \mathbf{F}_{i}^{tlak} = - \nabla p(\mathbf{x}_i) = - \sum_j m_j \frac{p_j}{\rho_j} \nabla W(|\mathbf{x}_i - \mathbf{x}_j|, h).
                   \end{equation}

                    \noindent Bohužel takto vypočítané síly nejsou symetrické (nesplňují Newtonův třetí zákon) \cite{Mueller2003}, tento problém je nejvíce patrný, když simulace obsahuje jen dvě částice. Protože gradient \emph{smoothing kernelu} je v středu, tj. na pozici  částice $i$, roven nule, částice použije pro výpočet jen tlak částice $j$. Müller řeší tento problém jednoduchým průměrovaním hodnot $p_i$ a $p_j$
                    
                    \begin{equation}
                        \mathbf{F}_{i}^{tlak} = - \nabla p(\mathbf{x}_i) = - \sum_j m_j \frac{p_i + p_j}{2 \rho_j} \nabla W(|\mathbf{x}_i - \mathbf{x}_j|, h).
                   \end{equation}
                   
                   Podobný problém nastává při výpočtu sil způsobené viskozitou tekutiny 
                   
                   \begin{equation}
                        \mathbf{F}_{i}^{viskozita} = \mu \nabla^2 \mathbf{v}(\mathbf{x}_i) = \mu \sum_j m_j \frac{\mathbf{v}_j}{\rho_j} \nabla^2 W(|\mathbf{x}_i - \mathbf{x}_j|, h),
                   \end{equation}

                    \noindent protože se rychlosti u každé částice liší. Symetrizace, kterou navrhl Müller \cite{Mueller2003}, spočívá v tom, že síla způsobena viskozitou zavisí pouze na rozdílech rychlostí částice od svých sousedů, což přirozeně vede k
                    
                    \begin{equation}
                        \mathbf{F}_{i}^{viskozita} = \mu \nabla^2 \mathbf{v} (\mathbf{x}_i) = \mu \sum_j m_j \frac{\mathbf{v}_j - \mathbf{v}_i}{\rho_j} \nabla^2 W(|\mathbf{x}_i - \mathbf{x}_j|, h),
                    \end{equation}
                    
                    Gravitace a externí síly jsou přímo aplikovány na samotné částice. Se všemi těmito informacemi lze vypočítat NSE, které lze formulovat Newtonovým druhým zákonem
                    
                    \begin{equation}
                        \mathbf{a}_i = \frac{ D \mathbf{v}_i }{ Dt } = \frac{\mathbf{F}_i}{ \rho_i },
                    \end{equation}
                    
                    \noindent kde $a_i$ je zrychlení, $\rho_i$ je hustota a $\mathbf{F}_i$ součet sil působící na částici $i$. Pro získání rychlosti $\mathbf{v}_i$ stačí integrovat podle času zrychlení $\mathbf{a}_i$. Müllerův simulátor \cite{Mueller2003} pro integraci používal algoritmus Leap--Frog \cite{Pozrikidis2008}, který integruje podle konstantní časových rozestupů, pro lepší výsledky navrhl algoritmy založené na Courant--Friedrichs--Lewy podmínce \cite{Desburn1996}.
                    
                    Posledním krokem je vizualizace částic. Bez žádných speciálních metod by částice měly pouhý tvar kuliček, které nejsou nijak spojeny. Řešením je tzv. \emph{point splatting} \cite{Zwicker2001}, který na základě množiny bodů vytvoří jednolitý útvar.
                    
                    
                    \begin{figure}\centering
                        \includegraphics[width=0.5\textwidth]{img/sph_density_curve}
                        \caption{Interpolace pole hustoty podle SPH \cite{SPHNvidia2011}}
                    \end{figure}
                    
                    \begin{figure}\centering
                        \includegraphics[width=0.5\textwidth]{img/sph_vis}
                        \caption{Müllerova simulace vody založená na SPH \cite{Mueller2003}}
                    \end{figure}
                    
                %TODO PSEUDOCODE
                    
                \subsection{Hybridní metody}
                    Předešlé dvě metody představují extrémními případy pro simulaci vodního hladiny. Procedulární metody nereagují na okolní prostředí, ale jsou výpočetně rychlé. Částicové systémy na druhou stranu počítají s dynamickým prostředím, ale jsou výpočetně náročné. Hybridní metody jsou kombinací dvou, které zároveň zjednodušují problematiku vodní hladiny jako procedulární metody a zachovávají některé fyzikální vlastnosti.
                    
                    Pro simulaci větších vodních ploch jsou částicové metody zbytečně náročné. Většina částic se bude nacházet pod hladinou a ve stojatých vodách se jejich pozice nebude výrazně měnit. Proto by bylo pro ně zbytečné počítat NSE.  Hybridní metody tuto výpočetní překážku řeší podobně jako procedulární, kdy problematiku vodní hladiny redukují z trojrozměrného prostoru do dvourozměrného \cite{Mueller2008}, transformují výšku vrcholů podle aproximací NSE.
                    
                    Nuttapongova hybridní metoda využívá \emph{Shallow Water Equations} (SWE) \cite{Nuttapong2010}, soustava parciálních diferenciálních rovnic, které zjednodušují NSE do 2D výškových map. Real-time výpočet SWE je ale stále drahá operace. Bridson zjednodušil její integraci výměnou za stabilitu simulace \cite{Nuttapong2010}.  
                    
                    Další variantou je modelování hladiny podle vlnových rovnic \cite{Bridson2007}, hyperbolická parciální diferenciální rovnice. Vlnové rovnice popisují řadu vln od vlnění struny kytary až po vlnění vodní hladiny.
                    
                    \subsubsection{Simulace podle vlnové rovnice}
                        Vlnová rovnice nahlíží na vodní hladinu jako na napnutou elastickou membránu, která nebere v potaz, co se pod ní děje. Protože problematiku redukuje do dvourozměrného prostoru, místo NSE využívá pro fyzikální korektní pohyb pouhý Newtonův druhý zákon. 
                        
                        Vlnová rovnice a síla, která vyvolává pohyb, jsou v následující části vyvozené. Pro jednoduchost je problém redukován na 1D, kde vypadá vlna jako napnutá vibrující struna (viz. obrázek \ref{fig:wave_eq}), postup lze analogicky převést do 2D. Jednodimenzionální vlnová rovnice předpokládá, že jsou splněny tyto podmínky \cite{WaveEqDerivation}:
                        \begin{enumerate} \label{en:wave_as}
                         \item hmotnost na jednotku délky je konstantní, 
                         \item struna je perfektně elestická a neklade neodpor ohybu,
                         \item gravitační síla je zanedbatelná (dominantní silou je ta, která strunu napíná),
                         \item výchylky struny v horizontální ose, ose x, jsou zanedbatelné (struna se hýbe jen ve vertikální ose y),
                         \item vertikální výchylky jsou malé a sklony struny od osy x jsou velmi malé (v obrázku \ref{fig:wave_eq} úhly $\theta$ a $\theta \Delta  \theta$).
                        \end{enumerate}
                        
                        \begin{figure}\centering
                            \includegraphics[width=\textwidth]{img/waveeq}
                            \caption{Vlna v 1D: (a) vibrujicí struna, (b) zvětšení segmentu $AB$ \cite{WaveEq2022}} \label{fig:wave_eq}
                        \end{figure}
                        
                        Nechť funkce $u(x,t)$ udává, jak je struna na pozici $x$ v čase $t$ vychýlená od osy x . V případě, že je struna v klidové stavu, struna splývá s osou x. Dále nechť na segment struny $AB$, část sturny od $x$ do $x + \Delta x$, kde $\Delta x \xrightarrow{} 0 $, působí napínací síly $F_{T_1}$ a $F_{T_2}$. Za překlodu, že se segement $AB$ málo vychyluje od osy x, lze jejich velikosti sil považovat za identické, 
                        $F_{T_1} = F_{T_2} = F_T$ \cite{Lewin2008}. Kvůli čtvrtému předpokladu \ref{en:wave_as} se horizontální síly vyruší a stačí spočítat síly působící ve vertikální ose.
                        
                        \begin{equation}
                            F_{vert} = -F_T \sin(\theta) + F_T \sin(\theta + \Delta \theta)
                        \end{equation}
                        
                        \noindent Síla $-F_T \sin(\theta)$ působí u bodu $A$ směrem dolů a síla $F_T \sin(\theta + \Delta \theta)$ u bodu $B$ opačným směrem. Díky pátemu předpokladu \ref{en:wave_as} lze aproximovat $\sin(\theta)$ a $\sin(\theta + \Delta \theta)$ jako velikost jejich úhlu.
                        
                        \begin{equation}
                            F_{vert} = -F_T \theta + F_T (\theta + \Delta \theta) = F_T \Delta \theta
                        \end{equation}
                        
                        \noindent Po získání síly působící na segment $AB$ lze aplikovat Newtonův druhý zákon.
                        
                        \begin{equation}
                            F_{vert} = ma \\
                        \end{equation}
                        \begin{equation} \label{eq:wave_newton}
                            F_{T} \Delta \theta = (\Delta x \mu) \frac{\partial^2 u}{\partial t^2} \\
                        \end{equation}
                        
                        \noindent $\mu$ značí hmotnost na jednotku délky. Zrychlení $a$ lze přepsat jako druhou derivací funkce $u$ podle $t$. Dále pro $ \Delta x \xrightarrow{} 0$ platí, že
                        
                        \begin{equation} \label{eq:tg}
                            \tg(\theta) = \frac{\partial u}{\partial x}
                        \end{equation}
                        
                        \noindent a pro derivaci celé rovnice \ref{eq:tg} podle x
                        
                        \begin{equation} \label{eq:tg_dx}
                            \frac{1}{\cos^2(\theta)} \frac{d \theta}{dx} = \frac{\partial^2 u}{\partial x^2}.
                        \end{equation}

                        \noindent Pro aproximaci hodnoty $\cos^2(\theta)$ se opět užije pátého předpokladu, že sklony $\theta$ a $\theta + \Delta \theta$ jsou velmi malé, podle kterého platí $\cos(\theta) \approx \cos(\theta + \Delta \theta) \approx 1$. Následně protože $x \xrightarrow{} 0$, lze přepsat rovnici \ref{eq:tg_dx} jako 
                        
                        \begin{equation}
                            \frac{\Delta \theta}{\Delta x} = \frac{\partial^2 u}{\partial x^2}
                        \end{equation}
                        
                        \begin{equation} \label{eq:tg_dx_final}
                            \Delta \theta = \Delta x \frac{\partial^2 u}{\partial^2 x}.
                        \end{equation}
                        
                        \noindent Po dosazení rovnice \ref{eq:tg_dx_final} do rovnice \ref{eq:wave_newton}, vydělení hodnotou $\Delta x$ a prohození členů získáme 1D vlnovou rovnici 
                        
                        \begin{equation} 
                             \frac{\partial^2 u}{\partial t^2} = \frac{F_T}{\mu} \frac{\partial^2 u}{\partial x^2}.
                        \end{equation}
                        
                        %% TODO CHECK REFRENCES
                        \noindent Pro zdůraznění $\frac{F_T}{\mu}$ pozitvní hodnoty se v literatuře uvádí vlnová rovnice jako 
                        
                        \begin{equation} \label{eq:1d_wave_eq}
                             \frac{\partial^2 u}{\partial t^2} = c^2 \frac{\partial^2 u}{\partial x^2}.
                        \end{equation}
                        
                        \noindent Ve slově rovnice \ref{eq:1d_wave_eq} popisuje, že zrychlení funkce $u$ je přímě uměrné změně jejího sklonu. V případě, že   
                        
                        
                        
                        
                        
                        

                        
                        

                        


                    
                    
                    
                


