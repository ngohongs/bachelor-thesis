 \begin{introduction}
	Kvalita herních titulů za poslední let ohromně vzrostla. Jejich úspěch lze připsat nejen zajímavému příběhovému obsahu, ale také i jejich vizuálnímu zpracování. Právě v grafickém provedení můžeme vidět největší pokrok. Díky hardwarovým zlepšení ve výpočetních technologií zejména v grafických kartách se obraz grafických aplikací čím dál více blíží fotorealistkým výsledkům a s příchodem zařízení pro virtuální a rozšiřenou realitu je o realistý obraz ještě větší zájem.
	
	Co rozlišuje scény napříč historií her jsou speciální efekty. Na základě kvality jejich provedení je hráč hlouběji vnořen do virtuálního prostředí a následně i do příběhu. Jedním z efektů jsou přírodní fenoména jako např. pohyb plamene, vlnění hladiny vody, proudění větru\ldots{} Fyzikálně korektní chování přírodních jevů však do dnešního dne zůstavá mezi komplexnější problémy.
	
	Mezi těmi výpočtně nejobtížnějšími je považováno proudění tekutin, do kterých patří jak plynné skupenství, tak i kapalné. Pro vizualici se většina metod zabývá zejména kapalinami kvůli její povaze (jsou lidským okem vidět), ale v některých případech lze postupy pro simulování proudění kapalin využít i pro zobrazení plynů jako např. kouře nebo plamene.
	
	Kvůli vysoce dynamickému chování tekutin se však v současnosti nejrealističtějších výsledků dosáhne hlavně off-line metodami. Real-time aplikace využívají stejných principů, ale s podstatnými kompromisi jak paměťovými, tak i výpočetnímí. Vzhledem k tomu, že některé aplikace nevyužijí přesné a hlavně výpočetně drahé simulace, tak napodobují jen výsledné efekty, které se nijak neopírají o fyzikální pravidla. Ačkoli off-line metody zobrazují věruhodně chování kapalin, resp. plynů, nejsou nijak zastoupeny v hrách, neboť interakce s uživatelem a dynamické prostředí scén je nedovolí využít.
	
    Obsah této bakalářské práce se zaobírá real-time simulací vodního povrchu a jeho případnými světelnými efekty jak na hladině, tak i na tělesa pod ní, jejíž aplikace by bylo možné využít k fotorealistkému zobrazení virtuálních scén v hrách.
\end{introduction}
