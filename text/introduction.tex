 \begin{introduction}
	Kvalita herních titulů za posledních let výrazně vzrostla. Jejich úspěch lze připsat nejen zajímavému příběhovému obsahu, ale také i jejich vizuálnímu zpracování. Právě v grafickém provedení můžeme vidět největší pokrok. Díky hardwarovým zlepšení v grafických kartách se obraz her čím dál více blíží fotorealistickým výsledkům a s příchodem zařízení pro virtuální a rozšířenou realitu je o realistický obraz ještě větší zájem.
	
	Co rozlišuje scény napříč historií her, jsou speciální efekty. Na základě kvality jejich provedení je hráč hlouběji vnořen do virtuálního světa a následně i\,do jeho příběhu. Jedním z takových efektů jsou např. přírodní fenoména jako pohyb plamene, vlnění hladiny vody, proudění větru\ldots{}
	
	Fyzikálně korektní chování přírodních jevů však zůstává do dnešního dne mezi komplexnějšími problémy. Mezi těmi výpočetně nejobtížnějšími je považováno proudění tekutin, do kterých patří jak plynné, tak i kapalné skupenství. Pro vizualizaci se většina metod zabývá zejména kapalinami kvůli její povaze (jsou lidským okem viditelné), ale v některých případech lze postupy pro simulování proudění kapalin využít i pro zobrazení plynů jako např. kouře nebo plamene.
	
	Kvůli vysoce dynamickému chování tekutin se však v současnosti nejrealističtějších výsledků dosáhne hlavně \emph{off-line} metodami. \emph{Real-time} aplikace využívají stejných principů, ale s podstatnými kompromisy jak paměťovými, tak i výpočetními. Vzhledem k tomu, že některé aplikace nevyužijí přesné a hlavně výpočetně drahé simulace, tak napodobují jen výsledné efekty, které se nijak neopírají o fyzikální zákony. Ačkoli \emph{off-line} metody zobrazují věrohodně chování kapalin, resp. plynů, nejsou nijak zastoupeny v hrách, neboť je interakce s uživatelem a dynamické prostředí scén nedovolí využít.
	
    Protože vývojáři her často ve svých produktech vodní hladiny a ostatní dynamické přírodní fenoména ručně animují a protože se jejich simulaci kvůli výpočetní náročnosti spíše vyhýbají, zkoumám ve své bakalářské práci možné metody simulace vodních ploch, které lze za pomocí současného hardwaru spočítat v reálném čase. Obsah této bakalářské práce se zabývá kromě \emph{real-time} simulace vodního povrchu, také jeho případnými světelnými efekty jak na hladině, tak i na tělesa pod ní, jejíž aplikace by bylo možné využít k\;realistickému zobrazení virtuálních scén v hrách.
    
    Bakalářská práce je následovně členěna. Nejdříve je čtenář krátce seznámen s mechanikou tekutin, která je nezbytnou prerekvizitou pro pochopení simulace vodních útvarů. Následovně jsou probrány dynamické a optické vlastnosti vodních ploch, k jejímž hlavním rysům jsou uvedeny algoritmy, které je simulují. V další části je probrán návrh a implementace simulátoru vodní hladiny podle vybraných metod a technologií probraných v analýze. Na závěr je provedeno vizuální vyhodnocení použitých metod.
    
\end{introduction}
