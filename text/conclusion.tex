 
\begin{conclusion}
	Jedním z cílů práce bylo provést analýzu současných možností pro simulaci vodního povrchu. Rozbor obsahuje několik metod založených na různých přístupech k simulaci kapalin. Na základě typu scén her a míry interakce vodní plochy s prostředím jsou jednotlivé metody různě vhodné. 
	
	Kromě analýzy samotných metod simulace je proveden i rozbor aktuálních technologií pro \emph{real-time} renderované aplikace. Jsou v něm probrány jak grafické API, tak i herní \emph{enginy}.
    
    Dalším cílem bylo vytvořit testovací scénu s vybraným algoritmem z analýzy  pro simulaci vodní plochy. Moji volbou byla metoda výpočtu vlnové rovnice a to z důvodu, že zahrnuje kladné vlastnosti jak procedulárních, tak i\;částicových metod simulace. Simulace podle vlnové rovnice zachovává výpočetní rychlost procedulárních metod a fyzikální korektnost částicových systémů. Optické vlastnosti vodní plochy jako odrazy, refrakce a kaustiky jsou implementovány pomocí zjednodušené metody vrhání paprsků, které za pomoci podpůrných textur aproximují průsečík světelných paprsků v prostoru obrazu. Barva vodní hladiny je dále spočítána podle Fresnelových rovnic, které míchají barvy odrazeného a refraktovaného světla.
    
    Testovací scéna je implementována za pomocí grafického API OpenGL, ale principy simulace jsou lehce přenositelné na ostatní nástroje. Vodní plocha v\;aplikaci reaguje opticky na prostředí testovací scény odrážením okolní scény, refraktováním obrazu nebo vrháním kaustik na podvodní objekty. Zachovává také jistou míru interakce, která umožňuje uživateli aplikace rozpohybovat vodní hladinu.
    
    Aktuálně simulátor vodní hladiny dokáže vykreslovat vodní plochu, jejíž tvar má obdélníkovou podobu. Protože tyto tvary lze vidět jen u umělých vodních útvarů, bylo by pro další iterace vývoje přínosné, aby simulátor dokázal simulovat i jiné tvary, čehož by nejspíše šlo dosáhnout pomocí masky nad výškovou mapou. Pro více realistické zobrazení hladiny by dále bylo třeba vylepšit odhad průsečíků světelných paprsků zejména pro výpočet reflekcí a refrakcí. 
\end{conclusion}
